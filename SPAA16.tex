% This is "sig-alternate.tex" V2.1 April 2013
% This file should be compiled with V2.5 of "sig-alternate.cls" May 2012
%
% This example file demonstrates the use of the 'sig-alternate.cls'
% V2.5 LaTeX2e document class file. It is for those submitting
% articles to ACM Conference Proceedings WHO DO NOT WISH TO
% STRICTLY ADHERE TO THE SIGS (PUBS-BOARD-ENDORSED) STYLE.
% The 'sig-alternate.cls' file will produce a similar-looking,
% albeit, 'tighter' paper resulting in, invariably, fewer pages.
%
% ----------------------------------------------------------------------------------------------------------------
% This .tex file (and associated .cls V2.5) produces:
%       1) The Permission Statement
%       2) The Conference (location) Info information
%       3) The Copyright Line with ACM data
%       4) NO page numbers
%
% as against the acm_proc_article-sp.cls file which
% DOES NOT produce 1) thru' 3) above.
%
% Using 'sig-alternate.cls' you have control, however, from within
% the source .tex file, over both the CopyrightYear
% (defaulted to 200X) and the ACM Copyright Data
% (defaulted to X-XXXXX-XX-X/XX/XX).
% e.g.
% \CopyrightYear{2007} will cause 2007 to appear in the copyright line.
% \crdata{0-12345-67-8/90/12} will cause 0-12345-67-8/90/12 to appear in the copyright line.
%
% ---------------------------------------------------------------------------------------------------------------
% This .tex source is an example which *does* use
% the .bib file (from which the .bbl file % is produced).
% REMEMBER HOWEVER: After having produced the .bbl file,
% and prior to final submission, you *NEED* to 'insert'
% your .bbl file into your source .tex file so as to provide
% ONE 'self-contained' source file.
%
% ================= IF YOU HAVE QUESTIONS =======================
% Questions regarding the SIGS styles, SIGS policies and
% procedures, Conferences etc. should be sent to
% Adrienne Griscti (griscti@acm.org)
%
% Technical questions _only_ to
% Gerald Murray (murray@hq.acm.org)
% ===============================================================
%
% For tracking purposes - this is V2.0 - May 2012

\documentclass{sig-alternate}
\usepackage{epsfig}
\usepackage{comment}
%\usepackage{subfigure}
\usepackage[pdftex]{color}
\usepackage{url}
\usepackage{listings}
\usepackage{textcomp}
\usepackage{gnuplot-lua-tikz}
\usepackage{amssymb}
\usepackage{amsmath}
%\usepackage{minted}
%\usepackage{bm}
\usepackage{paralist}
%\usepackage{ulem}
\usepackage{subcaption}
\usepackage{algorithm}
\usepackage{algpseudocode}
\usepackage{graphicx}
\usepackage{float}
\definecolor{lbcolor}{rgb}{0.9,0.9,0.9}

\lstset{
	backgroundcolor=\color{lbcolor},
		tabsize=4,
		rulecolor=,
		language=matlab,
		basicstyle=\scriptsize,
		upquote=true,
		aboveskip={1.5\baselineskip},
		columns=fixed,
		showstringspaces=false,
		extendedchars=true,
		breaklines=true,
		prebreak = \raisebox{0ex}[0ex][0ex]{\ensuremath{\hookleftarrow}},
		frame=single,
		showtabs=false,
		showspaces=false,
		showstringspaces=false,
		identifierstyle=\ttfamily,
		keywordstyle=\color[rgb]{0,0,1},
		commentstyle=\color[rgb]{0.133,0.545,0.133},
		stringstyle=\color[rgb]{0.627,0.126,0.941},
}

\newcommand{\rb}[1]{{\color{red}{\it RB: #1}}}
\newcommand{\dz}[1]{{\color{blue}{\it DZ: #1}}}
\newcommand{\jv}[1]{{\color{orange}{\it JV: #1}}}

\begin{document}

% Copyright
\setcopyright{acmcopyright}
%\setcopyright{acmlicensed}
%\setcopyright{rightsretained}
%\setcopyright{usgov}
%\setcopyright{usgovmixed}
%\setcopyright{cagov}
%\setcopyright{cagovmixed}


% DOI
%\doi{10.475/123_4}

% ISBN
%\isbn{123-4567-24-567/08/06}

%Conference
\conferenceinfo{SPAA '16}{June 27 - 29, 2016, Asilomar State Beach, California, USA}

%\acmPrice{\$15.00}

%
% --- Author Metadata here ---
%\conferenceinfo{WOODSTOCK}{'97 El Paso, Texas USA}
%\CopyrightYear{2007} % Allows default copyright year (20XX) to be over-ridden - IF NEED BE.
%\crdata{0-12345-67-8/90/01}  % Allows default copyright data (0-89791-88-6/97/05) to be over-ridden - IF NEED BE.
% --- End of Author Metadata ---

\title{Semi-External Memory Sparse Matrix Multiplication on Billion-node
	Graphs in a Multicore Architecture}
%\subtitle{[Extended Abstract]
%\titlenote{A full version of this paper is available as
%\textit{Author's Guide to Preparing ACM SIG Proceedings Using
%\LaTeX$2_\epsilon$\ and BibTeX} at
%\texttt{www.acm.org/eaddress.htm}}}
%
% You need the command \numberofauthors to handle the 'placement
% and alignment' of the authors beneath the title.
%
% For aesthetic reasons, we recommend 'three authors at a time'
% i.e. three 'name/affiliation blocks' be placed beneath the title.
%
% NOTE: You are NOT restricted in how many 'rows' of
% "name/affiliations" may appear. We just ask that you restrict
% the number of 'columns' to three.
%
% Because of the available 'opening page real-estate'
% we ask you to refrain from putting more than six authors
% (two rows with three columns) beneath the article title.
% More than six makes the first-page appear very cluttered indeed.
%
% Use the \alignauthor commands to handle the names
% and affiliations for an 'aesthetic maximum' of six authors.
% Add names, affiliations, addresses for
% the seventh etc. author(s) as the argument for the
% \additionalauthors command.
% These 'additional authors' will be output/set for you
% without further effort on your part as the last section in
% the body of your article BEFORE References or any Appendices.

\numberofauthors{4} %  in this sample file, there are a *total*
% of EIGHT authors. SIX appear on the 'first-page' (for formatting
% reasons) and the remaining two appear in the \additionalauthors section.
%

%for single author (just remove % characters)
%\author{
%{\rm Da Zheng, Randal Burns}\\
%Department of Computer Science \\
%Johns Hopkins University
%\and
%{\rm Joshua Vogelstein}\\
%Institute for Computational Medicine \\
%Department of Biomedical Engineering \\
%Johns Hopkins University
% copy the following lines to add more authors
%\and
%{\rm Alexander S. Szalay}\\
%Department of Physics and Astronomy \\
%Johns Hopkins University
%} % end author

\date{30 November 2015}
% Just remember to make sure that the TOTAL number of authors
% is the number that will appear on the first page PLUS the
% number that will appear in the \additionalauthors section.

\maketitle
\begin{abstract}
Due to random memory access, sparse matrix multiplication is traditionally
performed in memory and scales to large matrices using the distributed memory
of multiple nodes.
In contrast, we scale sparse matrix multiplication by utilizing commodity
SSDs. We implement sparse matrix dense matrix multiplication (SpMM) in
a semi-external memory (SEM) fashion, i.e., we keep the sparse matrix on SSDs
and dense matrices in memory. As such, our SEM SpMM can incorporate
with many in-memory optimizations for large sparse graphs with power-law
distribution in vertex degree and near-random vertex connection. Coupled with
many I/O optimizations, our SEM SpMM achieves performance
comparable to our in-memory implementation on a large parallel machine and
outperforms the implementations in Trilinos and Intel MKL.
Our experiments show that the SEM SpMM achieves almost 100\% performance
of the in-memory implementation on some graphs when the dense matrix has more
than four columns; it achieves at least 65\% performance of the in-memory
implementation for all of our graphs when the dense matrix has only one column.
We apply our SpMM to three important data analysis applications and show that
our SSD-based implementations can significantly outperform state of the art
of these applications.
\end{abstract}


%
% The code below should be generated by the tool at
% http://dl.acm.org/ccs.cfm
% Please copy and paste the code instead of the example below. 
%
%\begin{CCSXML}
%<ccs2012>
% <concept>
%  <concept_id>10010520.10010553.10010562</concept_id>
%  <concept_desc>Computer systems organization~Embedded systems</concept_desc>
%  <concept_significance>500</concept_significance>
% </concept>
% <concept>
%  <concept_id>10010520.10010575.10010755</concept_id>
%  <concept_desc>Computer systems organization~Redundancy</concept_desc>
%  <concept_significance>300</concept_significance>
% </concept>
% <concept>
%  <concept_id>10010520.10010553.10010554</concept_id>
%  <concept_desc>Computer systems organization~Robotics</concept_desc>
%  <concept_significance>100</concept_significance>
% </concept>
% <concept>
%  <concept_id>10003033.10003083.10003095</concept_id>
%  <concept_desc>Networks~Network reliability</concept_desc>
%  <concept_significance>100</concept_significance>
% </concept>
%</ccs2012>  
%\end{CCSXML}

%\ccsdesc[500]{Computer systems organization~Embedded systems}
%\ccsdesc[300]{Computer systems organization~Redundancy}
%\ccsdesc{Computer systems organization~Robotics}
%\ccsdesc[100]{Networks~Network reliability}


%
% End generated code
%

%
%  Use this command to print the description
%
\printccsdesc

% We no longer use \terms command
%\terms{Theory}

\keywords{Semi-external memory; SSDs; sparse matrix multiplication}

\section{Introduction}
% What problem are you going to solve.
Sparse matrix multiplication is very important computation with a wide variety
of applications in scientific computing, machine learning and data mining.
For example, many matrix factorization algorithms on a sparse matrix such as
singular value decomposition (SVD) \cite{svd} and non-negative matrix
factorization (NMF) \cite{nmf} requires sparse matrix multiplication.
Many graph analysis algorithms such as PageRank \cite{pagerank} can be
formulated as sparse matrix multiplication \cite{Mattson13}. Some of
the algorithms such as PageRank and SVD require sparse matrix vector
multiplication while some others such as NMF require sparse matrix dense
matrix multiplication.

% Why is it hard?

It is challenging to implement an efficient kernel of sparse matrix
multiplication, especially for many real-world graphs such as social networks
and Web graphs. Sparse matrix multiplication is notorious for achieving only
a small fraction of the peak performance of a modern processor \cite{Williams07}.
It becomes even more challenging to perform this operation on graphs due to
random memory access caused by near-random vertex connection and load imbalancing
caused by the power-law distribution in vertex degree. Many real-world graphs
are enormous. For example, Facebook's social network has billions of vertices
and today's web graphs are much larger. Furthermore, graphs cannot be
clustered or partitioned effectively \cite{leskovec} to localize access.
Therefore, sparse matrix multiplication on graphs is frequently the bottleneck
in an application.

%How have others addressed the problem?
Current research focuses on sparse matrix vector multiplication (SpMV) in memory
for small matrices and scaling to a large sparse matrix in a large cluster,
where the aggregate memory is sufficient to store the sparse matrix
\cite{Williams07, Yoo11, Boman2013}.
The distributed solution for sparse matrix multiplication leads to significant
network communication and is usually bottlenecked by the network.
As such, this operation requires fast network to achieve performance.
However, a supercomputer or a large cluster with fast network communication
is inaccessible or too expensive to many people.

\dz{can we demonstrate the distributed solution indeed has scalability problems?}

%What is the nature of your solution?

We explore an alternative solution that scales sparse matrix dense matrix
multiplication (SpMM) by utilizing commodity solid-state drives (SSDs) and
perform this operation in a semi-external memory (SEM) fashion
\cite{flashgraph, Abello98}. That is, we keep the sparse matrix on SSDs and
the dense matrix in memory. Even though SpMM can be implemented with SpMV,
we optimize SpMM directly to explore data locality in SpMM and reduce I/O
in semi-external memory. Given fast SSDs, we demonstrate that the SEM solution
can achieve performance comparable to state-of-art in-memory implementations
for sparse matrix multiplication while increasing the scalability in proportion
to the ratio of non-zero entries to rows or columns in a sparse matrix.
SSDs are energy-efficient storage media \cite{}. Thus, our solution introduces
an energy-efficient architecture for large-scale sparse matrix multiplication.

% Why is it new/different/special?

Although SSDs can deliver high IOPS and sequential I/O throughput, we have
to overcome many technical challenges to construct an SSD-friendly sparse matrix
multiplication kernal to achieve performance comparable to in-memory
counterparts. Even though SSDs have high IOPS, their sequential I/O throughput
is still significantly higher than random I/O and SSDs are an order of
magnitude slower than DRAM in throughput. Furthermore, random write is harmful
to SSDs \cite{sfs}. It increases write amplification, decreases I/O throughput
and shortens the lives of SSDs.

Semi-external memory provides a scalable SpMM
solution that is friendly for SSDs, parallelization and in-memory optimizations.
It streams the sparse matrix from SSDs for computation, which results in maximal
I/O throughput from SSDs. It streams the output matrix to SSDs if
memory is insufficient to store the output matrix, resulting in
the minimum amount of data written to SSDs and maximizing I/O throughput with
large I/O. While maximizing I/O throughput, we also compress
the sparse matrix to further accelerate retrieving the sparse matrix from SSDs.
In the parallel setting, each thread streams its own partitions for computation.
As such, the computation in each thread is independent to each other.
We deploy multiple in-memory optimizations specifically designed for power-law
graphs. For example, we assign partitions of the sparse matrix dynamically to
threads for load balancing, deploy cache blocking to increase CPU cache hits,
distribute the dense matrix to NUMA nodes to fully utilize the memory
bandwidth of a NUMA machine and organize the dense matrix in the row-major order
to explore data locality in SpMM.

Our semi-external memory solution adapts to machines with different memory
capacities. When the dense matrix is larger than memory, we split the dense
matrix vertically into multiple partitions so that each partition can fit in
memory. As such, the minimum memory requirement of our solution is $O(n)$,
where $n$ is the number of rows in the dense matrix. By keeping more columns
in the dense matrix in memory, we reduce I/O from SSDs in SpMM. When the number
of columns in a dense matrix increases, SEM SpMM becomes CPU bound, instead of
I/O bound on fast SSDs.

We further develop three important applications in scientific computing and
data mining with our SEM SpMM: PageRank \cite{pagerank}, eigendecomposition
\cite{} and non-negative matrix factorization \cite{nmf}. Each of the applications
requires SpMM with different numbers of columns in the dense matrix, which
results in different strategies of placing data in memory.
With the three applications, we demonstrate optimal data placement for
different memory capacities.

% What are it's key features?

Our result shows that for many real-world sparse graphs, our SEM sparse matrix
multiplication can achieve at least 60\% performance of its in-memory
implementations and significantly outperforms Trilinos \cite{trilinos} and
MKL \cite{mkl} on a large parallel machine with 48 CPU cores. As the number
of columns in
the dense matrix increases, the performance gap between the in-memory and
SEM implementations shrinks. When the dense matrix has more than four columns,
the system becomes CPU bottlenecked and the SEM solution achieves almost
100\% performance of the in-memory implementation for some of our graphs.
This suggests that the SEM solution requires sufficient memory to achieve
performance but additional memory cannot improve performance but waste energy.
We apply the SEM sparse matrix multiplication to some important
data analysis applications that require sparse matrix multiplication and compete
our solution with the state-of-art implementations of these applications.
\dz{show some brief performance results.} As such, we conclude that
semi-external memory coupled with SSDs delivers a fast and energy-efficient
solution for large-scale sparse matrix multiplication. It can also serves
as a building block and offers new design possibilities for large-scale
data analysis, replacing memory with larger, cheaper, more energy-efficient SSDs
and processing bigger problems on fewer machines.


\section{Related Work}
Recent sparse matrix multiplication studies focus on in-memory optimizations.
Williams et al.~\cite{Williams07} describe optimizations for sparse matrix
vector multiplication (SpMV) in multicore architectures. Yoo et al.~\cite{Yoo11}
and Boman et al.~\cite{Boman2013} optimize distributed SpMV for large
scale-free graphs with 2D partitioning to reduce communication between
machines. In contrast, Sparse matrix dense matrix multiplication (SpMM) receives
less attention from the high-performance computing community. Even though
SpMM can be implemented with SpMV, SpMV fails to explore data locality in
SpMM. Aktulga et al.~\cite{Aktulga14} optimize SpMM with cache blocking.
Koanantakool et al.~\cite{Koanantakool16} experiment different parallel
algorithms for sparse matrix dense matrix multiplication and analyze
their communication cost in distributed memory.
We advance SpMM with a focus on optimizations for semi-external memory.

Compressed row storage (CSR) and compressed column storage (CSC) formats are commonly
used sparse matrix formats in many numeric libraries such as Intel MKL \cite{mkl}
and Trilinos \cite{trilinos}. However, these formats are not designed for graphs.
Sparse matrix multiplication with these formats on graphs incurs many random memory
accesses. %Other, more modern sparse matrix formats have been designed.
Sparsity \cite{Im04} designs a format that encodes both register blocking and cache blocking to
increase data reuse in the CPU cache for sparse matrix multiplication. Register blocking
requires explicit storage of zero values in register blocks. This strategy
wastes space and computation for graphs because graphs are very
sparse and have nearly random vertex connection. Buluc et al.~\cite{Buluc08}
further advance sparse matrix format
by doubly compressed sparse column (DCSC) for hypersparse submatrices after 2D
partitioning on a sparse matrix. This format significantly reduces the storage
size of a 2D-partitioned sparse matrix. Our format further reduces the storage
size of a sparse matrix. SciDB \cite{Stonebraker11} developed a general purpose
dense array format that includes 2-d and 3-d partitioning and ghost cells
\cite{Soroush11}.  It is not well-suited to sparse matrixes.

%DZ TODO scidb cite -- http://www.odbms.org/wp-content/uploads/2014/04/The_Architecture_of_SciDB.pdf
% Array Store cite -- http://scidb.cs.washington.edu/paper/sigmod362-soroush.pdf

Abello et al.~\cite{Abello98} introduced the semi-external memory algorithmic
framework for graphs. Pearce et al.~\cite{Pearce10} implement several 
semi-external memory graph traversal algorithms for SSDs. FlashGraph
\cite{FlashGraph} adopted the concept and performs graph algorithms with
vertex state in memory and edge lists on SSDs. This work extends the semi-external
memory concept to matrix operations.

GridGraph \cite{gridgraph} is a general-purpose graph processing framework on
a single machine
and scales to large graphs using disks. It performs 2D partitioning on a graph
to reduce CPU cache misses in graph analysis and deploys 2-level hierarchical
partitioning to reduce I/O. This graph engine is designed for graph
algorithms expressed as sparse matrix vector multiplication and keeps both
graphs and vertex computation state on disks. In contrast, our work focuses on
optimizations on sparse matrix dense matrix multiplication and performs this
operation in semi-external memory to reduce I/O, especially to minimize writes
to SSDs. GridGraph runs much more slowly than our SEM SpMM implementation.
%\dz{Do we need to compare with it directly?}

MapReduce \cite{MapReduce} is also commonly used for processing large graphs.
PEGASUS \cite{pegasus} is a popular graph processing engine built on MapReduce
and expresses graph algorithms as a generalized form of sparse matrix-vector
multiplication. GBase \cite{gbase} is a graph database built on MapReduce.
It optimizes the graph storage for graph queries and analysis and perform graph
queries and analysis with sparse matrix vector multiplication.
Other work \cite{Liao14, Yin14, Liu10} implement non-negative matrix
factorization (NMF) with MapReduce. Although these MapReduce-based
implementations can scale to very large graphs, they run orders of
magnitude more slowly than optimized solutions, such as Trillinos and Intel MKL.

Zhou et al.~\cite{Zhou12} implement an LOBPCG \cite{Arbenz05} eigensolver in
an SSD cluster. Their implementation targets nuclear many-body Hamiltonian
matrices, which are much denser and have smaller dimensions than many sparse
graphs. Therefore, their solution stores the sparse matrix on SSDs and keep
the entire vector subspace in RAM. In contrast, our SEM-SpMM handles dense
matrices of different sizes and our eigensolver stores both the sparse matrix
and the vector subspace on SSDs.

Anasazi \cite{anasazi} is an eigensolver framework in the Trilinos project
\cite{trilinos}. This framework implements block extension of multiple
eigensolver algorithms such as Block Krylov-Schur \cite{krylovschur},
Block Davidson \cite{Arbenz05} and LOBPCG \cite{Arbenz05}. Anasazi uses
the matrix implementations in Trilinos that run in distributed memory.

Intel Math Kernel Library \cite{mkl} is an efficient and parallel linear
algebra library with matrix operations optimized for Intel
platforms. It provides an efficient sparse matrix multiplication
for regular sparse matrices. In contrast, our sparse matrix multiplication
optimizes for power-law graphs with near-random vertex connection.


\section{SAFS}

SAFS \cite{safs} is a user-space filesystem for a high-speed SSD array in
a NUMA (non-uniform memory architecture) machine. It is implemented as
a library and runs in the address space
of its application. It is deployed on top of the Linux native filesystem.
SAFS was originally designed for optimizing small I/O accesses. However,
sparse matrix multiplication and dense matrix operations
generate much fewer but much larger I/O. Therefore, we provide additional
optimizations to maximize sequential I/O throughput from a large SSD array.

The original SAFS has a dedicated I/O thread for each SSD. The I/O thread
accesses the SSD exclusively to avoid lock contention in the Linux kernel.
Application threads have to send I/O requests to one of the I/O threads
with message passing when accessing data from SSDs. It is necessary to have
one I/O thread for
an SSD when applications issue many small I/O accesses because processing
a large number of I/O accesses consumes a significant number of CPU cycles.
However, when the workload only has large I/O requests, each I/O request takes
much longer time to complete. As a result, the I/O threads are constantly put
into sleep while waiting for I/O and each I/O completion may suffer from
the latency of a thread context switch.

The latency of a thread context switch becomes noticeable on a high-speed SSD
array under a sequential I/O workload and it becomes critical to avoid thread
context switch to gain I/O performance. Therefore, instead of having an I/O
thread for each SSD, we use only a single I/O thread for each NUMA node, which
is responsible for all of the SSDs connected to the NUMA node. As such, an I/O
thread processes many more I/O requests to amortize the latency of a context
switch. Similarly, application threads communicate with I/O threads through
message passing when issuing I/O requests. If the computation in application
threads did not saturate CPU, SAFS would put the application threads into
sleep while they were waiting for I/O. This results in many thread context
switches and underutilization of both CPU and SSDs. To saturate I/O,
an application thread issues asynchronous I/O and poll for I/O to avoid thread
context switches after completing all computation available to it.

To better support access to many relatively small files simultaneously, SAFS
stripes data in a file across SSDs with a different striping order for each file.
This strategy stores data from multiple files evenly across SSDs and improves
I/O utialization. Due to the sequential I/O workload, FlashEigen stripes data
across SSDs with a large block size, on the order of megabytes, to increase I/O
throughput and potentially reduce write amplification on SSDs \cite{Tang15}.
Such a large block size may cause storage skew for small files
on a large SSD array if every file stripes data in the same order. Using
the same striping order for all files may also cause skew in I/O access.
Therefore, SAFS generates a random striping order for each file to evenly
distribute I/O among SSDs when a file is created. SAFS stores the striping
order with the file for future data retrieval.

\section{Sparse matrix multiplication}
Sparse matrix multiplication on graphs usually leads to many random memory
accesses and its performance is usually limited by the random memory performance
of DRAM. To scale sparse matrix multiplication to a sparse graph with billions
of vertices, we perform this operation in semi-external memory (SEM). That is,
we keep dense matrices in memory and the sparse
matrix on SSDs. This strategy enables nearly in-memory performance while achieving
the scalability in proportion to the ratio of edges to vertices in a graph.

\subsection{The sparse matrix format}
The state-of-art numeric libraries store a sparse matrix in compressed row storage
(CSR) or compressed column storage (CSC) format. However, these formats incur
many CPU cache misses in sparse matrix multiplication on many real-world graphs
due to their nearly random vertex connection. They also require a relatively
large storage size. For a graph with billions of edges, we have to use eight
bytes to store the row and column indices. For semi-external memory sparse
matrix multiplication, SSDs may become the bottleneck if a sparse matrix has
a large storage size.
Therefore, we need to use an alternative format for sparse matrices to increase
CPU cache hits and reduce the amount of data read from SSDs.

\begin{figure}
\centering
\includegraphics[scale=0.3]{./sparse_mat.pdf}
\caption{The format of a sparse matrix.}
\label{sparse_mat}
\end{figure}

To increase CPU cache hits, we deploy cache blocking \cite{Im04} and store
non-zero entries of a sparse matrix in tiles (Figure \ref{sparse_mat}).
When a tile is small, the rows from the input and output dense matrices
involved in the multiplication with the tile are always kept in the CPU cache
during the multiplication. The optimal tile size should fill the CPU cache
with the rows of the dense matrices involved in the multiplication with
the tile and is affected by the number of columns of the dense matrices,
which is chosen by users. Instead of generating a sparse matrix with
different tile sizes optimized for different numbers of columns in the dense
matrices, we use a relatively small tile size and rely on the runtime system
to optimize for different numbers of columns (in section \ref{sec:exec}).
In the semi-external memory, we expect that the dense matrices do not
have more than eight columns in sparse matrix multiplication. Therefore, we
use the tile size of $16K \times 16K$ by default to balance the matrix storage
size and the adaptibility to different numbers of columns.

\begin{figure}
\centering
\includegraphics[scale=0.5]{./tile_format.pdf}
\caption{The storage format of a tile in a sparse matrix.}
\label{tile_format}
\end{figure}

To reduce the overall storage size of a sparse matrix, we use a compact format
to store non-zero entries in a tile. In very sparse matrices such as
many real-world graphs, many rows in a tile do not have any non-zero entries.
The CSR (CSC) format requires an entry for each row (column) in the row
(column) index. Therefore, the CSR or CSC format wastes space when storing elements
in a tile. Instead, we only keep data for rows with non-zero entries in a tile
shown in Figure \ref{tile_format} and refer to this format as SCSR (Super
Compressed Row Storage). This format maintains a row header for each non-empty
row. A row header has an identifier to indicate the row number, followed by
column indices. 
The most significant bit of the identifier is always set to 1, while the most
significant bit of a column index entry is always set to 0. As such, we can easily
distinguish a row identifier from a column index entry and determine the end
of a row. Thanks to the small size of a tile, we use two bytes to further store a row
number and a column index entry to reduce the storage size. Since the most
significant bit is used to indicate the beginning of a row, this format allows
a maximum tile size of $32K \times 32K$.

For many real-world graphs, many rows in a tile have only one non-zero entry,
thanks to the sparsity of the graphs and nearly random vertex connection.
Iterating over single-entry rows requires to test the end of a row for every
non-zero entry, resulting in many extra conditional jump CPU instructions
in sparse matrix multiplication.
In contrast, the coordinate format (COO) is more suitable for storing these
single-entry rows. It does not increase the storage size but significantly
reduces the number of conditional jump instructions when we iterate
them. As a result, we hybrid SCSR and COO to store non-zero entries in a tile
with COO stored behind the row headers of SCSR. All non-zero entries are
stored together at the end of a tile.

We organize tiles in a sparse matrix in tile rows and maintain a matrix index
for them. Each entry of the index stores the location of a tile row on SSDs
to facilitate random access
to tile rows. This is useful for parallelizing sparse matrix multiplication.
Because a tile contains thousands of rows, the matrix index requires a very
small storage size even for a billion-node graph. We keep the entire index
in memory during sparse matrix multiplication.

\subsection{The dense matrices in sparse matrix multiplication}
Dense matrices in sparse matrix multiplication are tall-and-skinny matrices
with millions or even billions of rows but only a small number of columns.
The number of rows in a dense matrix is determined by the number of vertices
in a sparse graph and the number of columns is determined by applications.
The dense matrix is kept in memory for semi-external memory (SEM) sparse matrix
dense matrix multiplication (SpMM),
so the size of the input dense matrix governs memory consumption
of SpMM. Given the limited amount of RAM in a machine, the number of columns
in a dense matrix has to be small.

For a non-uniform memory architecture (NUMA), we partition the input dense matrix
horizontally and store partitions evenly across NUMA nodes to fully utilize
the bandwidth of memory and inter-processor links in sparse matrix
multiplication. The NUMA architecture is prevalent in today's multi-processor
servers, where each processor connects to its own memory banks. As shown in
Figure \ref{dense_mat} (a), we assign multiple
contiguous rows in a row interval to a partition, which is assigned to a NUMA
node. A row interval always has $2^i$ rows for efficiently locating a row
with bit operations. The row interval size is also multiple of the tile size of
a sparse matrix so that multiplication on a tile only needs to access rows
from a single row interval. The elements in the input dense matrix are stored
in row-major order to increase data locality in SpMM.

\begin{figure}
\centering
\includegraphics[scale=0.4]{./dense_matrix.pdf}
\caption{The data layout of tall-and-skinny (TAS) dense matrices. A TAS
dense matrix is partitioned horizontally into many row intervals.
(a) For an in-memory matrix, row intervals are stored across NUMA nodes and
elements are stored in row-major order; (b) for an SSD-based matrix, elements
inside a row interval are stored in column-major order.}
\label{dense_mat}
\end{figure}

\subsection{Execution of sparse matrix multiplication} \label{sec:exec}
We perform sparse matrix dense matrix multiplication in semi-external memory
and optimize it for different numbers of columns in the dense matrices.
Thanks to semi-external memory, sparse matrix multiplication streams data
in the sparse matrix from SSDs, which maximizes I/O throughput of the SSDs.

To better utilize CPU cache, we process tiles of a partition in
\textit{super tile}s (Figure \ref{sparse_mat}). The tile size of a sparse
matrix is specified when the sparse matrix image is created and is relatively
small to handle different numbers of columns in the dense matrices.
A \textit{super tile} is composed of tiles from multiple tile rows and its
size is determined at runtime by three factors: the number of columns
in the dense matrices, the CPU cache size and the number of threads that
share the CPU cache. An optimal size for a \textit{super tile} fills
the CPU cache with the rows from the dense matrices involved in
the computation with the \textit{super tile}.

We partition a sparse matrix horizontally for parallelization (Figure
\ref{sparse_mat}). When a thread gets tile rows, it reads them asynchronously
from SSDs and processes them completely independently. Once a partition
is ready in memory, the worker thread multiplies the partition with the input
dense matrix. A thread processes one \textit{super tile} at a time. It stores
the intermediate result in a buffer allocated in the local memory to reduce
remote memory access when it goes through all the tiles in the partition.

We maintain a global task queue for sparse matrix multiplication and a worker
thread gets one task at a time from the queue. A task may
indicates the computation on a \textit{super tile} row or a single tile row.
At the beginning of the computation, threads get larger tasks; when
the computation get close to the end, threads get smaller tasks. This strategy
achieves good load balancing. The other benefit of maintaining a global
task queue is to maintain the global execution order, which becomes essential
when the output dense matrix needs to be written to SSDs. When a thread gets
a task, the final computation result from the task may be small. Instead of
writing the computation result immediately whenever it is generated, we merge
computation results from multiple threads and write them with a single I/O.
As such, we need to maintain a global execution order to help us merge.

There are three options of keeping the output dense matrix of SpMM. In some
applications, we can keep the output dense matrix in memory if a machine has
sufficient memory. Otherwise, we write the output dense matrix to SSDs.
In this case, we stream the output dense matrix to SSDs when data in a portion
is generated. Horizontal partitioning ensures that the data written to SSDs is
always the final result of sparse matrix multiplication. Another option is to
stream the data to the subsequent operations when data is ready. \dz{I need to
implement this.} As such, we minimize the data written to SSDs.

In spite of nearly random edge connection in a real-world graph,
there exists regularity that allows vectorization to improve performance
in sparse matrix dense matrix multiplication. For each non-zero entry, we
need to multiply it with the corresponding row from the input dense matrix
and add the result to the corresponding row in the output dense matrix.
These operations can be accomplished by the vector CPU instructions such as
AVX \cite{avx}. The current implementation relies on GCC's auto-vectorization
to translate the C code to the vector CPU instructions by predefining the matrix
width in the code.

When accessing a sparse matrix on SSDs, we keep a set of memory buffers for
I/O access to reduce the overhead of memory allocation.
For a large spare matrix, each tile row is fairly large, on the order
of tens of megabytes. The operating system usually allocate a memory buffer
for such an I/O size with \textit{mmap()} and populates the buffer with physical
pages when the buffer is used. It is computationally expensive to populate
large memory buffers frequently. When accessing high-throughput I/O devices,
such overhead can cause substantial performance loss. Therefore, we keep a set
of memory buffers allocated previously and reuse them for new I/O requests.
Because tile rows in a sparse matrix usually have differnt sizes, we resize
a previously allocated memory buffer if it is too small for a new I/O request.

In some applications such as non-negative matrix factorization, the dense matrices
involved in sparse matrix multiplication may not fit in memory. We partition
the dense matrix vertically so that each partition has complete columns and can
fit in memory. We also organize each partition in the row-major order to increase
data locality. As such, we perform sparse matrix multiplication multiple times
to compute the final result. This approach requires $\lceil \frac{D}{M} \rceil$
passes over the sparse matrix. When RAM becomes the bottleneck, it does not
really matter.

\subsection{Caching in sparse matrix multiplication}
In the hardware where memory capacity exceeds the storage size of a vector, we
can use additional memory to cache a portion of the sparse matrix if sparse matrix
multiplication is required in an iterative algorithm. We cannot rely on
the page cache in SAFS \cite{sa-cache} to buffer some portion of the sparse matrix
because streaming a sparse matrix to memory always evict existing data in the page
cache and generates zero cache hits. Therefore, we explicitly cache some portion
of the sparse matrix in sparse matrix multiplication.

\subsection{The impact of the memory size on I/O in semi-external memory}
The memory size has significantly impact on I/O in semi-external memory.
The minimum memory requirement for semi-external memory sparse matrix
multiplication is $n \times c + \epsilon$, where $n$ is the number of rows or
columns of the sparse matrix, $c$ is the size of an element in the dense matrix,
and $\epsilon$ is the buffer size for part of the sparse matrix and the output
dense matrix.

When the input dense matrix cannot fit in memory, we should use the existing memory
to keep as many columns in the input dense matrix in memory as possible. Although
we can cache part of the sparse matrix, keeping more columns in memory saves more
I/O than using the same RAM to cache the sparse matrix. If all memory is used to
store the input dense matrixThe amount of data read from SSDs is
$\lceil \frac{n \times c \times k}{M} \rceil \times S$, where $k$ is the number
of columns in the dense matrix, $M$ is the size of memory, $S$ is the storage
size of the sparse matrix. In contrast, if all memory is used to cache the sparse
matrix, the amount of data read from SSDs is $[S - (M - n \times c)] \times k$.
$[S - (M - n \times c)] \times k > \lceil \frac{n \times c \times k}{M} \rceil \times S$,
if $S > M$ and $M > n \times c$.

Our strategy results in the minimum amount of data written to SSDs. When the output
dense matrix cannot fit in memory, the maximum write is $n \times c \times k$.
In other words, the output matrix only needs to be written to SSDs at most once.
The additional memory in the system can be used to buffer part of the output
dense matrix to reduce the amount of data written to SSDs.

As the number of columns kept in memory increases, the bottleneck of the system
may switch. When we can keep only one column of the input dense matrix in memory,
the system may be I/O bound; when we can keep more columns of the dense matrix
in memory, the system will become CPU or memory-bound. When the system becomes
CPU or memory-bound, the I/O complexity becomes irrelevant. The additional memory
should be used for buffering the result of sparse matrix multiplication.

\section{Applications}
We apply sparse matrix multiplication to three important applications widely
used in data mining and machine learning: PageRank \cite{pagerank},
eigendecomposition \cite{anasazi} and non-negative matrix factorization \cite{nmf}.
Each application requires a slightly different strategy of using memory in
sparse matrix multiplication.

\subsection{PageRank}
PageRank is an algorithm to rank the Web pages by using hyperlinks between Web
pages. It was first used by Google and is identified as one of the top 10 data
mining algorithms \cite{top10}. The algorithm runs iteratively and its update
rule for each Web pages in each iteration is
$PR(u) = \frac{1-d}{N} + d(\sum\limits_{v \in B(u)} \frac{PR(v)}{L(v)})$,
where $B(u)$ denotes the neighbor list of vertex $u$ and $L(v)$ denotes
the out-degree of vertex $v$. The PageRank algorithm can be expressed as sparse
matrix multiplication with the code below.

\vspace{-5pt}
\begin{minted}[mathescape, fontsize=\scriptsize,]{r}
pr2 <- (1-d)/N+d*A%*%(pr1/deg)
\end{minted}
\vspace{-5pt}

Based on the memory size, we can place different data in memory to reduce I/O.
When the memory can only fit a single vector, each iteration needs
to write a vector to SSDs and read two vectors (the result from
the previous iteration and the degree vector) and the sparse matrix
from SSDs.
When the memory can fit two vectors, the output vector can be kept
in memory, so each iteration needs to read the sparse matrix and only
one vector.
As more memory can be used, we can further keep the degree vector
and even cache part of the sparse matrix.

\subsection{Eigendecomposition}
Eigendecomposition and singular value decomposition (SVD) is commonly used
in many scientific fields as well as machine learning and data mining. Many
algorithms \cite{Lanczos, IRLM, krylovschur} and frameworks
\cite{arpack, anasazi, slepc} have been developed to solve a large eigenvalue
problem.

We take advantage of the Anasazi eigensolver framework \cite{anasazi} and
replace its original matrix operations with our semi-external memory sparse
matrix multiplication and external-memory dense matrix operations. To compute
eigenvalues of a $n \times n$ matrix, many eigenvalue algorithms for a large
sparse matrix require to construct a vector subspace with a sequence of
sparse matrix multiplication. Each vector in the subspace has the length of $n$.
For many sparse graphs such as social networks, the vector subspace requires
substantial storage size. Therefore, we need to keep part of the subspace
on SSDs. Eigensolvers perform some dense matrix operations on the subspace.
For example, eigensolvers need to orthogonalize the vectors in the subspace,
which requires dense matrix multiplication. The Anasazi eigensolvers have
block extension to update multiple
vectors in the subspace simultaneously and thus require sparse matrix dense
matrix multiplication. The most efficient Anasazi eigensolver on sparse graphs
is the KrylovSchur eigensolver \cite{krylovschur}, which updates a small number
of vectors (1-4) in the subspace simultaneously. Zheng et al.
\cite{flasheigen} provides the details of extending the Anasazi eigensolver
with external-memory operations.

The choice of data placement for an eigensolver is a little different from
PageRank. If a machine has more memory, the memory should be used to keep
		the input and output dense matrices. The dense matrices involved in
		sparse matrix multiplication have a small number of columns in
		the KrylovSchur eigensolver and usually can fit in memory. 
Additional memory should be used to buffer the vectors in the subspace
		to reduce I/O for dense matrix operations.

\subsection{Non-negative matrix factorization}
Non-negative matrix factorization (NMF) \cite{nmf} is to find two non-negative
low-rank matrices $W$ and $H$ to approximate a matrix $A \approx WH$. NMF is
typically used to factorize sparse matrices. NMF has many applications in
machine learning
and data mining. A well-known example is collaborative filtering \cite{cf} in
recommender systems. NMF is also applied to graphs, for example, for community
detection \cite{yang13, wang11}.

Many algorithms were designed to solve NMF and here we describe an algorithm
\cite{nmf} that requires a sequence of sparse matrix multiplication.
The algorithm use multiplicative update rules and update matrices $W$ and $H$
alternately. When updating $H$, the algorithm fixes $W$ and updates $H$.
Similarly, the algorithm fixes $H$ to update $W$.

$H_{a\mu} \leftarrow H_{a\mu} \frac{{(W^TA)}_{a\mu}}{{(W^TWH)}_{a\mu}}$,
$W_{ia} \leftarrow W_{ia} \frac{{(AH^T)}_{ia}}{{(WHH^T)}_{ia}}$

When applying NMF to a very sparse graph, the non-negative matrices $W$ and $H$
may not completely fit in memory. As such, we apply semi-external memory sparse
matrix multiplication to the algorithm above differently based on the memory size
and the number of columns in $W$ and $H$. If the memory is not large enough to
keep $W$ and $H$, we need to partition $W$ and $H$ vertically and run multiple
sparse matrix multiplications to compute $W^TA$ and $AH^T$. The choices of data
placement for NMF are shown as follows.
When memory is small, all memory should be used to keep as many
columns in the input dense matrix as possible to reduce I/O.
When the memory size is sufficiently large to cause memory to be
the bottleneck, additional memory should be used to buffer the output
matrix.


\section{Experimental Evaluation}

We evaluate the performance of our SEM-SpMM on multiple real-world billion-scale
graphs including a web-page graph with 3.4 billion vertices. We first measure
the performance of SEM-SpMM and compare it with our in-memory implementation
(IM-SpMM), which is simply the SEM-SpMM implementation with the sparse matrix
in memory.
We also compare SEM-SpMM with state-of-the-art in-memory implementations in
Intel MKL (\textit{mkl\_dcsrmm}) and Trilinos Tpetra. We use Intel MKL 2015
and Trilinos v12.0.1 for the experiments. We demonstrate the effectiveness of
CPU and I/O optimizations on SEM-SpMM and evaluate the overall performance
of the applications in Section \ref{sec:spmm:apps}.

We conduct experiments on a non-uniform memory architecture machine with
four Intel Xeon E7-4860 processors, clocked at 2.6 GHz, and 1TB memory of
DDR3-1600. Each processor has 12 cores. The machine has three LSI SAS 9300-8e
host bus adapters (HBA) connected to a SuperMicro storage chassis, in which
24 OCZ Intrepid 3000 SSDs are installed. The 24 SSDs together are capable of
delivering 12 GB/s for read and 10 GB/s for write at maximum. The machine runs
Linux kernel v3.13.0. We use 48 threads for all implementations.

\begin{table}
\begin{center}
\footnotesize
\begin{tabular}{|c|c|c|c|c|}
\hline
Graph datasets & \# Vertices & \# Edges & Directed \\
\hline
Twitter \cite{twitter} & $42$M & $1.5$B & Yes \\
\hline
Friendster \cite{friendster} & $65$M & $1.7$B & No \\
\hline
%KNN graph \cite{} & $65$M & $6.5$B & No \\
%\hline
Page graph \cite{web_graph} & $3.4$B & $129$B & Yes \\
\hline
RMAT-40 \cite{rmat} & 100M & 3.7B & Yes \& No \\
\hline
RMAT-160 \cite{rmat} & 100M & 14B & Yes \& No \\
\hline
\end{tabular}
\normalsize
\end{center}
\caption{Graph data sets. We construct a directed and undirected version for
both RMAT-40 and RMAT-160.}
\label{graphs}
\end{table}

We use the adjacency matrices of the graphs in Table \ref{graphs} for performance
evaluation. The smallest graph we use has 42 million vertices and 1.5 billion
edges. The largest graph is the Page graph with 3.4 billion vertices
and 129 billion edges, which is two orders of magnitude larger than the smallest
graphs. We generate two synthetic graphs with R-Mat \cite{rmat} to fill the size
gap between the smallest and largest graph. We construct a directed and
undirected version for each of the synthetic graphs because some applications
in Section \ref{sec:spmm:apps} run on directed graphs and others run on undirected
graphs. The real-world datasets are publically available and the synthetic
datasets are generated with the RMAT implementation in the \textit{boost}
library\footnote{We use the parameters of $a=0.57$, $b=0.19$, $c=0.19$,
$d=0.05$.}. We always use the undirected version of the synthetic graphs for
the performance evaluation of sparse matrix multiplication. The Page graph is
clustered by domain. The MKL and Tpetra implementations cannot run on the Page
graph because its size exceeds the memory capacity of our NUMA machine.

\subsection{SEM-SpMM vs. IM-SpMM}

\begin{figure}
	\footnotesize
	\centering
	\begin{subfigure}[b]{0.5\textwidth}
		\centering
		\includegraphics[scale=1]{SpMM_figs/spmm_im_vs_sem.eps}
		\vspace{-5pt}
		\caption{The runtime performance of SEM-SpMM, normalized to IM-SpMM
		for the dense matrix with the same number of columns.}
		\label{perf:spmm_comp}
	\end{subfigure}
	\begin{subfigure}[b]{0.5\textwidth}
		\centering
		\vspace{5pt}
		\includegraphics[scale=1]{SpMM_figs/spmm_IO.eps}
		\vspace{-5pt}
		\caption{The I/O throughput generated by SEM-SpMM.}
		\label{perf:spmm_IO}
	\end{subfigure}
	\vspace{3pt}
	\caption{The performance and the I/O throughput of SEM-SpMM with dense
	matrices of different numbers of columns.}
	\label{perf:spmm_IM_vs_SEM}
\end{figure}

We compare the performance of SEM-SpMM with IM-SpMM to investigate
the performance penalty of scaling SpMM with SSDs. In this case, the dense
matrices have a small number of columns and are stored in memory. 

There is only a small performance penalty for semi-external memory (Figure
\ref{perf:spmm_comp}). The performance gap between IM-SpMM and SEM-SpMM
is affected by randomness of vertex connection. The gap is smaller if
vertex connection in a graph is more random. The Page graph is relatively
well clustered, so SpMM on this graph is less CPU-bound than others.
Even for the Page graph, SEM-SpMM gets 65\% performance of IM-SpMM.
The other factor of affecting the performance gap is the number of columns
in the dense matrices. The gap gets smaller as the number of columns in
the dense matrices increases.

We further measure the average I/O throughput generated by SEM-SpMM to indicate
the bottleneck of the system (Figure \ref{perf:spmm_IO}). SpMV on the Page
graph saturates the I/O bandwidth of SSDs and is clearly bottlenecked by I/O.
SpMV on other graphs (except Twitter) also generates high I/O throughput,
which consumes memory bandwidth and potentially interferes with random memory
access in SpMV. SEM-SpMV on the Twitter graph takes about 0.5 second to
complete and,
thus, startup overhead has significant impact on the average I/O throughput.
As the number of columns in the dense matrix increases, SpMM becomes more CPU
bound. For all graphs, SEM-SpMM requires a very small number of columns to
become CPU-bound and achieve close to 100\% performance of IM-SpMM.

Multiple factors affect the performance
gap between IM-SpMM and SEM-SpMM (Figure \ref{perf:spmm_sbm}). We illustrate
some of the factors with stochastic block model (SBM) \cite{sbm}, a random graph
generation model widely used in the graph community. In this experiment, we
generate SBM graphs with 100 million vertices and 3 billion edges.
One of the main factors is whether vertices in a graph are ordered
based on its cluster structures. If the vertices in a graph are randomly ordered,
SpMV on the graph generates many random memory access. Thus, the system is
bottlenecked by memory bandwidth and the performance gap between IM-SpMV and
SEM-SpMV is small. Because it is usually expensive to find the cluster structures
of real-world graphs and order vertices accordingly, it is a common case to run
SpMM on an unclustered graph. Even on a graph with vertices ordered based on
cluster structures, some cluster properties such as the number of clusters and
the ratio of edges inside and outside clusters have significant impact on
the performance gap.
As the number of clusters increases, the size of clusters gets smaller and
there are fewer random memory accesses inside clusters. Similarly, more
edges inside clusters also lead to fewer random memory accesses.

\begin{figure}
	\begin{center}
		\footnotesize
		\includegraphics[scale=1]{SpMM_figs/spmm-sbm.eps}
		\caption{The relative performance of SEM-SpMV on graphs generated with
			stochastic block model, normalized to IM-SpMV on the same graphs.
			IN/OUT indicates the ratio of edges inside and outside clusters.
			A graph can have vertices ordered based on the cluster structures
		(clustered) and randomly ordered (unclustered).}
		\label{perf:spmm_sbm}
	\end{center}
\end{figure}

\subsection{SEM-SpMM vs. other in-memory SpMM}
In this section, we compare SEM-SpMM with the Intel MKL and Trilinos Tpetra
implementations. Intel MKL runs on shared-memory machines. Trilinos Tpetra can run in
both shared memory and distributed memory, so we measure its performance in
our 48-core NUMA machine as well as an EC2 cluster. We run Tpetra in the largest
EC2 instances r3.8xlarge, where each has 16 physical CPU cores and 244GB of RAM
and is optimized for memory-intensive applications. The EC2 instances are
connected with 10Gbps network in the same placement group.

\begin{figure}
	\footnotesize
	\centering
	\begin{subfigure}[b]{0.5\textwidth}
		\centering
		\includegraphics[scale=1]{SpMM_figs/SpMV-awesomer.eps}
		\vspace{-5pt}
		\caption{SpMV}
		\label{perf:spmv}
	\end{subfigure}
	\begin{subfigure}[b]{0.5\textwidth}
		\centering
		\includegraphics[scale=1]{SpMM_figs/SpMM-awesomer.eps}
		\vspace{-5pt}
		\caption{SpMM with a dense matrix of 8 columns.}
		\label{perf:spmm8}
	\end{subfigure}
	\vspace{3pt}
	\caption{The performance of different sparse matrix multiplication
		implementations on the 48-core machine normalized to IM-SpMM for
	the same graphs.}
	\label{perf:spmm}
\end{figure}

Our SEM-SpMM significantly outperforms Intel MKL and Trilinos Tpetra on the natural
graphs on our NUMA machine (Figure \ref{perf:spmm}). In this case, we compare
performance of our SEM-SpMM with Intel MKL and Trilinos Tpetra for both sparse matrix
vector multiplication (SpMV) and sparse matrix dense matrix multiplication (SpMM).
The Tpetra implementation is optimized for SpMV. Our SEM-SpMM still constantly
outperforms Tpetra by a factor of $2-3$ even for SpMV. The MKL implementation has
better optimizations for SpMM than Trilinos Tpetra. Our SEM-SpMM is still almost
twice as fast as MKL in SpMM with a dense matrix of eight columns. There are multiple
reasons that our SEM-SpMM outperforms the ones in MKL and Tpetra. Neither MKL nor
Tpetra balance loads dynamically. In addition, they store a sparse matrix in CSC
or CSR format, which leads to many CPU cache misses. Because the SpMM in Tpetra
is implemented with MPI, it has additional memory copy overhead to exchange
partitions of the input dense matrix among processes.

SEM-SpMM only consumes a small fraction of memory compared with IM-SpMM and
other SpMM implementations (Figure \ref{perf:spmm_mem}). SEM-SpMM consumes
memory for the input dense matrix as well as per-thread local memory buffers
for the sparse matrix and the output dense matrix. When we use 48 threads for
SpMM, the memory used by local memory buffers in each thread is significant
but is relatively constant for different graph sizes. We only show the memory
consumption on one of the large graphs RMAT-160 in Figure \ref{perf:spmm_mem}.
Despite considerable memory used by local memory buffers, SEM-SpMM uses about
one tenth of the memory
used by IM-SpMM. We also observe that IM-SpMM consumes much less memory than
MKL and Tpetra owing to its compact format for sparse matrices.

\begin{figure}
	\begin{center}
		\footnotesize
		\includegraphics[scale=1]{SpMM_figs/SpMM-mem.eps}
		\caption{Memory consumption of different SpMM implementations on
		RMAT-160.}
		\label{perf:spmm_mem}
	\end{center}
\end{figure}

\begin{figure}
	\footnotesize
	\centering
	\begin{subfigure}[b]{0.5\textwidth}
		\centering
		\includegraphics[scale=1]{SpMM_figs/SpMV-EC2.eps}
		\vspace{-5pt}
		\caption{SpMV}
		\label{perf:ec2:spmv}
	\end{subfigure}
	\begin{subfigure}[b]{0.5\textwidth}
		\centering
		\includegraphics[scale=1]{SpMM_figs/SpMM-EC2.eps}
		\vspace{-5pt}
		\caption{SpMM with a dense matrix of 8 columns.}
		\label{perf:ec2:spmm8}
	\end{subfigure}
	\vspace{3pt}
	\caption{The performance of SEM-SpMM on our 48-core machine (SEM) and
		Trilinos Tpetra on EC2 clusters (2xEC2, 4xEC2, 8xEC2 and 16xEC2),
		normalized to IM-SpMM on our 48-core machine for the same graphs.
		We also show the performance of IM-SpMM on
	one of the EC2 instance (IM-EC2) where Trilinos Tpetra runs.}
	\label{perf:ec2}
\end{figure}

Our SpMM implementation uses much less computation resources to achieve
comparable performance and, in many cases, outperforms Trilinos Tpetra that
runs in the Amazon cloud, especially on real-world graphs (Figure
\ref{perf:ec2}). In this experiment, we run our SpMM implementation on both
our NUMA machine with 48 CPU cores
and one of the EC2 machines with 16 CPU cores. Owing to the compact format
for a sparse matrix, our SpMM implementation can run on all of the graphs
in memory on an EC2 instance. When Tpetra runs on 16 EC2 instances, it has
5 times as many CPU cores as our NUMA machine. Tpetra is not
able to run SpMV on RMAT-160 on two EC2 nodes. Even though an EC2 instance
has only 16 physical CPU cores, our IM-SpMM on an EC2 instance achieves around
half of the performance of our IM-SpMM on our NUMA machine. In contrast,
Trilinos Tpetra uses many more computation resources and still barely reaches
the same performance as our IM-SpMM and SEM-SpMM on our NUMA machine. One of
the main reasons that our SpMM implementation performs much
better on real-world graphs is that these graphs are more likely to cause
load imbalance. Our SpMM implementation balances load much better than
distributed implementations that partition data.

\subsection{SEM-SpMM with a large input dense matrix}

We further measure the performance of SEM-SpMM with a large input dense matrix,
in which neither the sparse matrix nor the dense matrices can fit in memory.
In this experiment, we measure the performance of multiplying a sparse matrix
with a dense matrix of 32 columns and the input dense matrix is stored on SSDs
initially. We study the impact of memory size on the performance of SEM-SpMM
by artificially varying the number of columns that can fit in memory. SEM-SpMM
accesses
the sparse matrix with direct I/O and, thus, varying the number of columns
in the dense matrix that fit in memory does not affect data access to the
sparse matrix. In each run, we need to load the input dense matrix from
SSDs and stream the output dense matrix to SSDs. We do not show the result on
the Page graph because the dense matrix with 32 columns for the Page graph
cannot fit in memory.

\begin{figure}
	\begin{center}
		\footnotesize
		\includegraphics[scale=1]{SpMM_figs/spmm-32cols.eps}
		\caption{The performance of SEM-SpMM with a dense matrix of 32 columns
			relative to IM-SpMM, when the number of columns of the input dense
		matrix kept in memory varies.}
		\label{perf:spmm32}
	\end{center}
\end{figure}

As more columns in the input dense matrix can fit in memory, the performance
of SEM-SpMM constantly increases (Figure \ref{perf:spmm32}). When the memory
can fit over four columns of the input dense matrix, SEM-SpMM gets over 50\%
of the performance of IM-SpMM. Even when only one column of the input dense
matrix can fit in memory, SEM-SpMM still gets 25\% of the in-memory performance.
When the entire input dense matrix can fit in memory, we get about 80\% of
the in-memory performance.

\begin{figure}
	\begin{center}
		\footnotesize
		\includegraphics[scale=1]{SpMM_figs/spmm-32cols-overhead.eps}
		\caption{The overhead breakdown of SEM-SpMM on the Friendster
			graph with a dense matrix of 32 columns when the number
		of columns in the input dense matrix kept in memory varies. }
		\label{perf:spmm32_over}
	\end{center}
\end{figure}

Two main factors lead to performance loss in SEM-SpMM when the input dense matrix
cannot fit in memory. We illustrate the contribution of four potential overheads
in SEM-SpMM on the Friendster graph (Figure \ref{perf:spmm32_over}). The main
performance loss comes from the loss of data locality in SpMM caused by
vertical partitioning of the input dense matrix (Vert-part). Partitioning
the dense matrix into one-column matrices contributes 60\% of performance loss.
It drops quickly when the vertical
partition size increases. Keeping the sparse matrix on SSDs (SpM-EM)
also contributes some performance loss when the dense matrix is partitioned
into small matrices. The overhead almost goes away when more than four columns
of the dense matrix can fit in memory. The overhead of streaming the output dense
matrix to SSDs (Out-EM) and reading the input dense matrix to memory (In-EM)
is less significant and remains the same for different memory sizes.

\subsection{Optimizations on sparse matrix multiplication}
Accelerating SEM-SpMM requires both computation and I/O optimizations.
We first evaluate the effectiveness of computation optimizations by deploying
them on IM-SpMM. We further show the effectiveness of I/O optimizations by
deploying them on SEM-SpMM with all computation optimizations.

Here we illustrate the most significant computation optimizations from Section
\ref{sec:spmm}. We start with an in-memory implementation that
performs sparse matrix multiplication on a sparse matrix in the CSR format
and apply the optimizations incrementally in the following order:
\begin{itemize} \itemsep1pt \parskip0pt \parsep0pt
	\item dispatch partitions of a sparse matrix to threads dynamically
		to balance load (\textit{Load balance}),
	\item partition dense matrices for NUMA (\textit{NUMA}),
	\item organize the non-zero entries in a sparse matrix into tiles to
		increase CPU cache hits (\textit{Cache blocking}),
	\item use CPU vectorization instructions to accelerate arithmetic
		computation (\textit{Vec}),
\end{itemize}

All of these optimizations have positive effects on sparse matrix
multiplication and all optimizations together speed up SpMM by $3-5$ times
(Figure \ref{perf:spmm_opt}). The degree of effectiveness
varies between different graphs and different numbers of columns in
the dense matrices. The largest performance boost is from cache blocking,
especially for SpMV. This is expected because these graphs have near-random
vertex connection, which leads many random memory access in sparse matrix
multiplication. Cache blocking significantly increases CPU cache hits to reduce
random memory access. CPU vectorization is only effective on SpMM because
it optimizes computation on a row of the dense matrix.
%For example, the NUMA optimization is more effective when
%the dense matrices have more columns because more columns in the dense
%matrices require more memory bandwidth. Cache blocking is very effective when
%the dense matrices have fewer columns because it can effectively increase CPU
%cache hits. When there are more columns in the dense matrices, data locality
%improves and the effectiveness of cache blocking becomes less noticeable.
%When there are too many columns, the rows from the input and output matrices
%can no longer be in the CPU cache.
With all optimizations, we have a fast in-memory implementation for both
sparse matrix vector multiplication and sparse matrix dense matrix multiplication.

\begin{figure}
	\begin{center}
		\footnotesize
		\includegraphics[scale=1]{SpMM_figs/SpMM_optimize.eps}
		\caption{The speedup of each computation optimization for SpMV and SpMM
			on the Friendster (F) and Twitter (T) graph. The input dense matrix
		in SpMM has 8 columns.}
		\label{perf:spmm_opt}
	\end{center}
\end{figure}

We evaluate I/O optimizations on SEM-SpMV against a base implementation that
has all of the computation optimizations and use doubly compressed sparse row
format (DCSR) to store tiles of a sparse matrix. We illustrate their
effectiveness on the Friendster graph and the Page graph. The first one
represents a graph that is not well clustered; the other one is clustered with
domain names. We apply the I/O optimizations in the following order:
\begin{itemize} \itemsep1pt \parskip0pt \parsep0pt
	\item use SCSR to reduce data read from SSDs (SCSR),
	\item reduce memory allocation overhead for I/O with per-thread buffer
		pools (\textit{buf-pool}),
	\item reduce the number of thread context switches for I/O accesses with I/O
		polling (\textit{IO-poll}),
\end{itemize}

\begin{figure}
	\begin{center}
		\footnotesize
		\includegraphics[scale=1]{SpMM_figs/io_opts.eps}
		\caption{The speedup of I/O optimizations for SpMV on the Friendster
		graph and the Page graph.}
		\label{perf:spmm_opt_io}
	\end{center}
\end{figure}

The I/O optimizations lead to substantial speedup over the base implementation,
but behave very differently on these two graphs (Figure \ref{perf:spmm_opt_io}).
On the unclustered graph (Friendster), SCSR requires a much smaller storage size
than DCSR (Figure \ref{fig:storage}) and thus achieves significant
speedup. The Page graph, on the other hand, is well clustered and DCSR already
achieves a small storage size. SCSR further reduces the storage size, but is
less significant. SpMV on the Page graph has less random memory access and is
I/O-bound even on a large SSD array. \textit{Buf-pool} and \textit{IO-poll}
increases I/O throughput and, thus, improves performance. In contrast, SEM-SpMV
with the Friendster graph in the SCSR format already achieves almost 80\% of
IM-SpMV and, thus, further I/O optimizations have less noticeable speedup.

\subsection{Performance of the applications}

We evaluate the performance of our implementations of the applications in
Section \ref{sec:spmm:apps}. We show the effectiveness of additional memory for
these applications and compare their performance with state-of-the-art
implementations.

\subsubsection{PageRank}
We evaluate the performance of our SpMM-based PageRank implementation
(SpMM-PageRank). This implementation requires the input vector to be in memory,
but it is optional to keep the output vector and the degree vector in memory.
PageRank is a benchmarking graph algorithm implemented by many graph processing
frameworks. We compare the performance of SpMM-PageRank with state-of-the-art
implementations in FlashGraph \cite{flashgraph}, a semi-external memory graph
engine, and GraphLab Create, the next generation of PowerGraph \cite{powergraph}.
The PageRank implementation in FlashGraph computes
approximate PageRank values while SpMM-PageRank and GraphLab Create compute
exact PageRank values. We run GraphLab Create completely in memory and
FlashGraph in semi-external memory. GraphLab Create is not able to compute
PageRank on the Page graph. We use FlashGraph v0.3 and a trial version of
GraphLab Create v1.9.

SpMM-PageRank in memory and in semi-external memory both significantly outperform
the implementations in FlashGraph and GraphLab Create (Figure \ref{perf:pagerank}).
Our SpMM is highly optimized for both CPU
and I/O. Even though SpMM-PageRank performs more computation than FlashGraph,
it performs the computation much more efficiently and
reads less data from SSDs than FlashGraph. SpMM-PageRank and the implementation
in GraphLab create performs the same computation, but SpMM-PageRank
performs the computation much more efficiently.

The experiment results also show that keeping more vectors in memory has modest
performance improvement for SpMM-PageRank. As such, SpMM-PageRank only needs
to keep one vector in memory, which results in very small memory consumption.

\begin{figure}
	\begin{center}
		\footnotesize
		\includegraphics[scale=1]{SpMM_figs/pagerank.eps}
		\caption{The runtime of SpMM-PageRank in 30 iterations. The SEM
			implementation keeps different numbers of vectors in memory
			(SEM-1vec, SEC-2vec, SEM-3vec). We compare them with
		the implementations in FlashGraph and GraphLab Create.}
		\label{perf:pagerank}
	\end{center}
\end{figure}

\subsubsection{Eigensolver}

We evaluate the performance of our SEM KrylovSchur eigensolver and compare
its performance
with our in-memory eigensolver and the Trilinos KrylovSchur eigensolver.
Usually, spectral analysis only requires a very small number of
eigenvalues, so we compute eight eigenvalues in this experiment. We run
the eigensolvers on the smaller undirected graphs
in Table \ref{graphs}. To evaluate the scalability of the SEM eigensolver,
we compute singular value decomposition (SVD) on the Page graph. Among all of
the eigensolvers, only our SEM eigensolver is able to compute eigenvalues
on the Page graph.

\begin{figure}
	\begin{center}
		\footnotesize
		\includegraphics[scale=1]{SpMM_figs/eigen-runtime-8ev.eps}
		\caption{The runtime of our SEM KrylovSchur, our in-memory eigensolver
			and the Trilinos eigensolvers when computing eight
			eigenvalues. SEM-min keeps the entire vector subspace on SSDs and
		SEM-max keeps the entire vector subspace in memory.}
		\label{fig:eigen}
	\end{center}
\end{figure}

For computing 8 eigenvalues, our SEM eigensolver achieves performance
comparable to our in-memory eigensolver and the Trilinos eigensolver
and can scale to very large graphs (Figure \ref{fig:eigen}).
Unlike PageRank, an eigensolver has many more vector or dense matrix operations.
As such, the memory size has noticeable impact on performance.
For the setting with the minimum memory consumption, it has at least 45\%
performance of our in-memory eigensolver; when keeping the entire subspace
in memory, it has almost the same performance as our in-memory eigensolver.

\subsubsection{NMF}
We evaluate the performance of our NMF implementation (SEM-NMF) on the directed
graphs in Table \ref{graphs}. The dense matrices for NMF can be as large as
the sparse matrix. As such, we experiment with the effect of the memory size on
the performance of SEM-NMF by varying the number of columns in memory from
the dense matrices. We also compare the performance of SEM-NMF with
a high-performance NMF implementation SmallK \cite{SmallK}, built on top of
the numeric library Elemental \cite{elemental}. We factorize
each of the graphs into two $n \times k$ non-negative dense matrices and
we use $k=16$ because $16$ is the largest $k$ that SmallK supports for
the graphs in Table \ref{graphs}. We use SmallK v1.6 and Elemental v0.85.

\begin{figure}
	\begin{center}
		\footnotesize
		\includegraphics[scale=1]{SpMM_figs/NMF.eps}
		\caption{The runtime per iteration of SEM-NMF on directed graphs.
			We vary the number of columns in the dense matrices that are kept
			in memory to evaluate effect of the memory size on the performance
		of SEM-NMF.}
		\label{perf:NMF}
	\end{center}
\end{figure}

We significantly improve the performance of SEM-NMF by keeping more columns
of the input dense matrix in memory (Figure \ref{perf:NMF}). The performance
improvement is more significant when the number of columns that fit in memory
is small. When we keep eight columns of the input dense matrix in memory,
SEM-NMF achieves over 60\% of the performance of the in-memory implementation.

SEM-NMF significantly outperforms other NMF implementations in the literature.
SmallK is the closest competitor. We run the same NMF algorithm in SmallK and
SEM-NMF outperforms SmallK by a large factor on all graphs (Figure
\ref{perf:NMF}). There are many MapReduce implementations in
the literature \cite{Liao14, Yin14, Liu10}. They run on sparse
matrices with tens of millions of non-zero entries but generally take
one or two orders of magnitude more time than our SEM-NMF on the sparse matrices
with billions of non-zero entries.


\section{Conclusions}
We present an alternative solution of scaling sparse matrix multiplication
to large sparse matrices by utilizing commodity SSDs.
We perform sparse matrix multiplication in semi-external memory (SEM), in which
we keep the sparse matrix on SSDs and the dense matrices in memory. Semi-external
memory increases scalability in proportion to the ratio of non-zero entries
to rows or columns in a sparse matrix. It incorporates well with many memory
optimizations for sparse matrix
multiplication such as cache blocking and partitioning the dense matrix for
NUMA machines. We also deploy a set of I/O optimizations for high-speed SSDs
such as I/O polling and preallocating memory buffers for I/O.

Our SEM sparse matrix multiplication is able to achieve performance
comparable to its in-memory counterparts while significantly outperforming
the MKL and Trilinos implementations. Our SEM implementation achieves almost
100\% performance of the in-memory implementation on some graphs when
the dense matrices have more than four columns in some graphs and can fit
in memory. Even when the dense matrix has only one column, it achieves at least
60\% performance of its in-memory counterpart on different
graphs.

For a machine with insufficient memory
to keep the entire input dense matrix in memory, we partition the dense matrix
vertically and run semi-external memory sparse matrix multiplication multiple
times. Given a fast SSD array, the SEM sparse matrix multiplication becomes
bottlenecked by CPU, even when the number of columns in the dense matrix
is small. As such, as long as a machine has sufficient memory to contain
a small number of columns of the input dense matrix, we can achieve performance
comparable to the in-memory implementation for a dense matrix with an arbitrary
number of columns.

We apply our sparse matrix multiplication to three important applications:
PageRank, eigendecomposition and non-negative matrix factorization. We demonstrate
how additional memory should be used in semi-external memory in each application.
In summary, additional memory should always first be used for the input dense
matrix, and then the output dense matrix. If there exists additional memory,
we can further improve performance by caching the sparse matrix.

SSDs are energy-efficient storage media \cite{}. Thus, our solution introduces
an energy-efficient architecture for large-scale sparse matrix multiplication.


%ACKNOWLEDGMENTS are optional
\section{Acknowledgments}

%
% The following two commands are all you need in the
% initial runs of your .tex file to
% produce the bibliography for the citations in your paper.
\bibliographystyle{abbrv}
\bibliography{SPAA16}  % sigproc.bib is the name of the Bibliography in this case% You must have a proper ".bib" file
%  and remember to run:
% latex bibtex latex latex
% to resolve all references
%
% ACM needs 'a single self-contained file'!
%
%APPENDICES are optional
%\balancecolumns

\end{document}
