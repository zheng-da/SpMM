% What problem are you going to solve.
Sparse matrix multiplication is very important computation with a wide variety
of applications in scientific computing, machine learning and data mining.
For example, matrix factorization algorithms on a sparse matrix such as
singular value decomposition (SVD) \cite{svd} and non-negative matrix
factorization (NMF) \cite{nmf} requires sparse matrix multiplication.
Graph analysis algorithms such as PageRank \cite{pagerank} can be
formulated as sparse matrix multiplication \cite{Mattson13}. Some of
the algorithms, such as PageRank and SVD, require sparse matrix vector
multiplication. Others, such as NMF, require sparse matrix dense
matrix multiplication.

% Why is it hard?

It is challenging to implement an efficient kernel of sparse matrix
multiplication, especially for sparse matrices that encode real-world graphs
such as social networks
and Web graphs. Sparse matrix multiplication is notorious for achieving only
a small fraction of the peak performance of a modern processor \cite{Williams07}.
It becomes even more challenging to perform this operation on graphs due to
random memory access caused by near-random connection and load imbalancing
caused by the power-law distribution in vertex degree. Many real-world graphs
are enormous. For example, Facebook's social network has billions of vertices
and today's web graphs are much larger. Furthermore, graphs cannot be
clustered or partitioned effectively \cite{leskovec} to localize access.
Therefore, sparse matrix multiplication on graphs is frequently the bottleneck
in an application.

%How have others addressed the problem?
Current research focuses on sparse matrix vector multiplication (SpMV) in memory
for small matrices and scaling to a large sparse matrix in a large cluster,
where the aggregate memory is sufficient to store the sparse matrix
\cite{Williams07, Yoo11, Boman2013}.
The distributed solution for sparse matrix multiplication leads to significant
network communication and network is usually the bottleneck.
As such, this operation requires fast network to achieve performance.
A supercomputer or a large cluster with a fast network is inaccessible or
too expensive for many users.

\dz{can we demonstrate the distributed solution indeed has scalability problems?}

%What is the nature of your solution?

On the other hand, the current trend of hardware design is to scale up
a machine in high performance computing.
These machines typically have multiple processors with many CPU cores and
a large amount of memory. They are also equipped with fast flash
memory such as solid-state drives (SSDs) to further extend memory capacity.
This conforms to node design for supercomputers.

We explore a solution that scales sparse matrix dense matrix multiplication
(SpMM) on a multi-core machine with commodity solid-state drives (SSDs) and
perform this operation in a semi-external memory (SEM) fashion
\cite{flashgraph, Abello98}. That is, we keep the sparse matrix on SSDs and
the dense matrix in memory. \dz{How to define SEM more formally?}
To enable semi-external memory for sparse matrix multiplication, we assume
that the memory of a machine is sufficient to keep at least one column
of the input dense matrix but is insufficient to hold the sparse matrix
in memory. Even though SpMM can be implemented with SpMV,
we optimize SpMM directly to explore data locality in SpMM and reduce I/O
in semi-external memory. Given fast SSDs, we demonstrate that the SEM solution
uses the resource of a multi-core machine well and
can achieve performance comparable to state-of-art in-memory implementations
for sparse matrix multiplication while increasing the scalability in proportion
to the ratio of non-zero entries to rows or columns in a sparse matrix.

% Why is it new/different/special?

Although SSDs can deliver high IOPS and sequential I/O throughput, we have
to overcome many technical challenges to construct a sparse matrix
multiplication kernal to achieve performance comparable to in-memory
counterparts. Even though SSDs have high IOPS, their sequential I/O throughput
is still significantly higher than random I/O and SSDs are an order of
magnitude slower than DRAM in throughput. Furthermore, random writes are harmful
to SSDs \cite{sfs}. They increase write amplification, decrease I/O throughput,
increase latency and shorten the lives of SSDs.

Semi-external memory provides a scalable SpMM solution that incorporates well
with I/O access to SSDs, parallelization and in-memory optimizations.
It streams the sparse matrix from SSDs for computation, which results in maximal
I/O throughput from SSDs. It streams the output matrix to SSDs once if
memory is insufficient to store the output matrix, resulting in
the minimum amount of data written to SSDs and maximizing I/O throughput.
We compress the sparse matrix to further accelerate retrieving the sparse
matrix from SSDs. In the parallel setting, each thread streams its own partitions.
The computation in each thread is independent.
We deploy multiple in-memory optimizations specifically designed for power-law
graphs. For example, we assign partitions of the sparse matrix dynamically to
threads for load balancing, deploy cache blocking to increase CPU cache hits,
distribute the dense matrix to NUMA nodes to fully utilize the memory
bandwidth of a NUMA machine, and organize the dense matrix in the row-major order
to explore data locality in SpMM.

Our semi-external memory solution adapts to machines with different memory
capacities. When the dense matrix is larger than memory, we split the dense
matrix vertically into multiple partitions so that each partition can fit in
memory. As such, the minimum memory requirement of our solution is $O(n)$,
where $n$ is the number of rows in the dense matrix. By keeping more columns
in the dense matrix in memory, we reduce I/O from SSDs in SpMM. When the number
of columns in a dense matrix increases, SEM SpMM becomes CPU bound, instead of
I/O bound on fast SSDs.

We develop three important applications in scientific computing and
data mining with our SEM SpMM: PageRank \cite{pagerank}, eigendecomposition
\cite{} and non-negative matrix factorization \cite{nmf}. Each application
requires SpMM with different numbers of columns in the dense matrix, which
results in different strategies of placing data in memory.
With the three applications, we demonstrate optimal data placement for
different memory capacities.

% What are it's key features?

Our result shows that for many real-world sparse graphs, our SEM sparse matrix
multiplication can achieve almost 100\% performance of our in-memory implementation
on a large parallel machine with 48 CPU cores
when the dense matrix has more than four columns. Even for sparse matrix vector
multiplication, our SEM implementation achieves at least 60\% performance of
our in-memory implementation and significantly outperforms Trilinos \cite{trilinos}
and MKL \cite{mkl}. The applications implemented with our SpMM significantly
outperforms the state-of-art implementations of these applications. As such,
we conclude that semi-external memory coupled with SSDs delivers an efficient
solution for large-scale sparse matrix multiplication. It can also serves
as a building block and offers new design possibilities for large-scale
data analysis, replacing memory with larger, cheaper, more energy-efficient SSDs
and processing bigger problems on fewer machines.
